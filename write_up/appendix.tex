\appendix
\pagenumbering{arabic}
\renewcommand*{\thepage}{\thechapter-\arabic{page}}
\chapter{Codes}\label{c:c}
\section{Main Codes}
The public version does not contain the main codes because updated and generalised versions of them may become part of future \textsc{cfour}\cite{cfour} releases.
\subsection{Calling and Printing PES and Their Derivatives}\label{s:cpes}
\inputminted[ linenos=true,
			 numbersep=5pt,
			 fontfamily=tt,
			 fontsize = \codefont,
			 gobble=0,
			 frame=lines,
			 framesep=2mm,
			 xleftmargin=0mm,
			 xrightmargin=0mm]{fortran}
			 {\cpth pes.f08}

\subsection{Auxiliary Functions Module (Heaviside, Kronecker's Delta, RK4G, Random Seed Initialiser)}
\inputminted[ linenos=true,
			 numbersep=5pt,
			 fontfamily=tt,
			 fontsize = \codefont,
			 gobble=0,
			 frame=lines,
			 framesep=2mm,
			 xleftmargin=0mm,
			 xrightmargin=0mm]{fortran}
			 {\cpth functions.f08}
\begin{comment}
\subsection{Single Avoided Crossing}
\inputminted[ linenos=true,
			 numbersep=5pt,
			 fontfamily=tt,
			 fontsize = \codefont,
			 gobble=0,
			 frame=lines,
			 framesep=2mm,
			 xleftmargin=0mm,
			 xrightmargin=0mm]{fortran}
			 {\cpth sc.f08}

\subsection{Double Avoided Crossing}
\inputminted[ linenos=true,
			 numbersep=5pt,
			 fontfamily=tt,
			 fontsize = \codefont,
			 gobble=0,
			 frame=lines,
			 framesep=2mm,
			 xleftmargin=0mm,
			 xrightmargin=0mm]{fortran}
			 {\cpth dc.f08}

\subsection{Extended Coupling}
\inputminted[ linenos=true,
			 numbersep=5pt,
			 fontfamily=tt,
			 fontsize = \codefont,
			 gobble=0,
			 frame=lines,
			 framesep=2mm,
			 xleftmargin=0mm,
			 xrightmargin=0mm]{fortran}
			 {\cpth ec.f08}

\subsection{Spin-Boson Model}
\inputminted[ linenos=true,
			 numbersep=5pt,
			 fontfamily=tt,
			 fontsize = \codefont,
			 gobble=0,
			 frame=lines,
			 framesep=2mm,
			 xleftmargin=0mm,
			 xrightmargin=0mm]{fortran}
			 {\cpth sb.f08}
\end{comment}
\begin{comment}
\section{Gnuplot Scripts}
\subsection{Plot PES}
\inputminted[ %linenos=true,
			 numbersep=5pt,
			 fontfamily=tt,
			 fontsize = \codefont,
			 gobble=0,
			 frame=lines,
			 framesep=2mm,
			 xleftmargin=0mm,
			 xrightmargin=0mm]{gnuplot}
			 {\spth pes_script.plt}

\subsection{Plot IR Spectra}
\inputminted[ %linenos=true,
			 numbersep=5pt,
			 fontfamily=tt,
			 fontsize = \codefont,
			 gobble=0,
			 frame=lines,
			 framesep=2mm,
			 xleftmargin=0mm,
			 xrightmargin=0mm]{gnuplot}
			 {\spth ir_script.plt}
\end{comment}

\section{Python 3.4 Scripts}
\subsection{Plot PES}
\inputminted[ linenos=true,
			 numbersep=5pt,
			 fontfamily=tt,
			 fontsize = \codefont,
			 gobble=0,
			 frame=lines,
			 framesep=2mm,
			 xleftmargin=0mm,
			 xrightmargin=0mm]{python3}
			 {\spth pes_script.py}
\begin{comment}
\subsection{Plot IR Spectra}
\inputminted[ %linenos=true,
			 numbersep=5pt,
			 fontfamily=tt,
			 fontsize = \codefont,
			 gobble=0,
			 frame=lines,
			 framesep=2mm,
			 xleftmargin=0mm,
			 xrightmargin=0mm]{python3}
			 {\spth ir_script.py}
\end{comment}

\subsection{Plot Transition Probabilities}
\inputminted[ linenos=true,
			 numbersep=5pt,
			 fontfamily=tt,
			 fontsize = \codefont,
			 gobble=0,
			 frame=lines,
			 framesep=2mm,
			 xleftmargin=0mm,
			 xrightmargin=0mm]{python3}
			 {\spth prob_script.py}

\subsection{Plot Transition Probabilities (Large Integration Step)}
\inputminted[ linenos=true,
			 numbersep=5pt,
			 fontfamily=tt,
			 fontsize = \codefont,
			 gobble=0,
			 frame=lines,
			 framesep=2mm,
			 xleftmargin=0mm,
			 xrightmargin=0mm]{python3}
			 {\spth prob_script_lh.py}

\subsection{Plot Transition Probabilities for Parallel Calculations}
\inputminted[ linenos=true,
			 numbersep=5pt,
			 fontfamily=tt,
			 fontsize = \codefont,
			 gobble=0,
			 frame=lines,
			 framesep=2mm,
			 xleftmargin=0mm,
			 xrightmargin=0mm]{python3}
			 {\spth prob_script_parallel.py}

\subsection{MPI Data Generator Script}
\inputminted[ linenos=true,
			 numbersep=5pt,
			 fontfamily=tt,
			 fontsize = \codefont,
			 gobble=0,
			 frame=lines,
			 framesep=2mm,
			 xleftmargin=0mm,
			 xrightmargin=0mm]{python3}
			 {\spth mpi_dat_gen.py}

\subsection{Plot MPI Data Script}
\inputminted[ linenos=true,
			 numbersep=5pt,
			 fontfamily=tt,
			 fontsize = \codefont,
			 gobble=0,
			 frame=lines,
			 framesep=2mm,
			 xleftmargin=0mm,
			 xrightmargin=0mm]{python3}
			 {\spth mpi_script.py}
%
\newpage
\setcounter{page}{1}
\chapter{Proofs}
\section{Transformation From Action-Angle to Cartesian Coordiantes}\label{s:aatocar}
Starting from \cref{eq:trans},
\begin{subequations}\label{eq:trans}
\begin{align}
H(\bm{P},\bm{R},\bm{n},\bm{q}) &= \frac{\bm{P}^{2}}{2 \mu} + \sum\limits_{k=1}^{F} n_{k} H_{kk}(\bm{R}) \label{eq:aahama}\\
& + 2 \sum\limits_{k<k'=1}^{F} \sqrt{(n_{k}+\gamma)(n_{k'}+\gamma)} \times \cos(q_{k}-q_{k'})H_{kk'}(\bm{R})~,\nnum
x_{k} &= \sqrt{2(n_{k} + \gamma)} \cos(q_{k})\label{eq:tran1}\\
p_{k} &= -\sqrt{2(n_{k} + \gamma)} \sin(q_{k})\label{eq:tran2}\\
n_{k} &= \frac{1}{2} p_{k}^{2} + \frac{1}{2} x_{k}^{2} - \gamma\label{eq:nk}~,
\end{align}
\end{subequations}
We first turn to the trigonometric term in \cref{eq:aahama}. We identify the trigonometric identity,
\begin{align}
\cos(\alpha-\beta) = \cos(\alpha)\cos(\beta) + \sin(\alpha)\sin(\beta)~,
\end{align}
and expand the angle-difference in the summation and substitute \cref{eq:tran1,eq:tran2},
\begin{subequations}
\begin{align}
& 2 \sum\limits_{k<k'=1}^{F} \sqrt{(n_{k}+\gamma)(n_{k'}+\gamma)} (\cos(q_{k})\cos(q_{k'}) + \sin(q_{k})\sin(q_{k'}))H_{kk'}\\
&= \sum\limits_{k<k'=1}^{F} \left\{\begin{aligned}
\sqrt{2(n_{k} + \gamma)} \cos(q_{k}) &\cdot \sqrt{2(n_{k'} + \gamma)} \cos(q_{k'})+\\
\left(-\sqrt{2(n_{k} + \gamma)} \sin(q_{k})\right) &\cdot \left(-\sqrt{2(n_{k'} + \gamma)} \sin(q_{k'})\right)
\end{aligned}\right\}H_{kk'}\\
& = \sum\limits_{k<k'=1}^{F} (x_{k}x_{k'} + p_{k}p_{k'})H_{kk'}
\end{align}
\end{subequations}
Which means we now have:
\begin{align}
H(\bm{P},\bm{R},\bm{n},\bm{q}) &= \frac{\bm{P}^{2}}{2 \mu} + \sum\limits_{k=1}^{F} n_{k} H_{kk}(\bm{R}) + \sum\limits_{k<k'=1}^{F} (p_{k}p_{k'} + x_{k}x_{k'})H_{kk'}~.
\end{align}

We then turn our attention to the middle term,
\begin{align}
\sum\limits_{k=1}^{F} n_{k} H_{kk}(\bm{R})~.
\end{align}
In order to have conservation of probability we have the normalisation condition,
\begin{align}\label{eq:norm}
\sum\limits_{k=1}^{F} n_{k} = 1~.
\end{align}
We then define the mean of the trace of the diabatic matrix $ \bar{H} $,
\begin{align}\label{eq:mean}
\bar{H} = \frac{1}{F}\sum\limits_{k=1}^{F} H_{kk}~.
\end{align}
This means we can write,
\begin{align}\label{eq:hequiv}
\sum\limits_{k=1}^{F} n_{k} H_{kk} \rightarrow \bar{H} + \sum\limits_{k=1}^{F} n_{k} (H_{kk} - \bar{H})~.
\end{align}
We can verify that \cref{eq:hequiv} is true by expanding the summation and using the normalisation condition in \cref{eq:norm},
\begin{align}
\sum\limits_{k=1}^{F} n_{k} H_{kk} & = \bar{H} + \sum\limits_{k=1}^{F} n_{k} (H_{kk} - \bar{H}) \Rightarrow \bar{H} + \sum\limits_{k=1}^{F} n_{k} H_{kk} - \sum\limits_{k=1}^{F} n_{k} \bar{H}\nnum
& \Rightarrow \bar{H} + \sum\limits_{k=1}^{F} n_{k} H_{kk} - \underbrace{(n_{1} + n_{2} +\dots + n_{F})}_{=1} \cdot \bar{H} \Rightarrow \sum\limits_{k=1}^{F} n_{k} H_{kk}~.
\end{align}
Now we recall \cref{eq:mean,eq:hequiv},
\begin{align}
\sum\limits_{k=1}^{F} n_{k} (H_{kk} - \bar{H}) = \sum\limits_{k=1}^{F} n_{k} (H_{kk} - \frac{1}{F} \sum\limits_{k=1}^{F}H_{kk})~.
\end{align}
By expanding the right hand side we find,
\begin{subequations}
\begin{align}
& n_{1}\left[H_{11} - \frac{1}{F} \left(H_{11} + H_{22} + \dots + H_{FF} \right) \right] + \nnum 
& n_{2}\left[H_{22} - \frac{1}{F} \left(H_{11} + H_{22} + \dots + H_{FF} \right) \right] + \dots + \\
& n_{F}\left[H_{FF} - \frac{1}{F} \left(H_{11} + H_{22} + \dots + H_{FF} \right) \right]~. \nonumber
\end{align}
\end{subequations}
Which can be rearranged to,
\begin{align}\label{eq:rear}
\frac{1}{F} \left\{\begin{aligned}
& H_{11} \left[ (F-1) n_{1} - n_{2} - \dots - n_{F} \right] + \\
& H_{22} \left[ - n_{1} + (F-1) n_{2} - \dots - n_{F} \right] + \dots + \\
& H_{FF} \left[ - n_{1} - n_{2} -\dots + (F-1) n_{F} \right]
\end{aligned}\right\}~.
\end{align}
We then take a look at a summation with a different structure,
\begin{align}\label{eq:altsum}
\frac{1}{F} \sum\limits_{k<k'=1}^{F} (n_{k}-n_{k'})(H_{kk}-H_{k'k'})~,
\end{align}
where the term $ k<k'=1 $ means that $ k $ starts at $ 1 $ and is always smaller than $ k' $. \Cref{t:iter} shows the summation's behaviour.
\begin{table}[H]
\centering
\caption[Behaviour of $ \sum\limits_{k<k'=1}^{F} kk'$.]{Behaviour of $ \sum\limits_{k<k'=1}^{F} kk'$.}
\label{t:iter}
\begin{tabular}{ccccccc}
\hline
\multirow{2}{*}{Iteration}&\multicolumn{5}{c}{Value}\\
\cline{2-7}
& $ 1 $ & $ 2 $ & $ 3 $ & $ \dots $ & $ F-1 $ & $ F $\\
\hline
$ 1 $ & $ k $ & $ k' $ & $ k' $ & $ \dots $ & $ k' $ & $ k' $ \\
$ 2 $ & - & $ k $ & $ k' $ & $ \dots $ & $ k' $ & $ k' $ \\
$ 3 $ & - & - & $ k $ & $ \dots $ & $ k' $ & $ k' $ \\
$ \vdots $ & $ \vdots $ & $ \vdots $ & $ \vdots $ & $ \ddots $ & $ \vdots $ & $ \vdots $ \\
$ F-1 $ & - & - & - & $ \dots $ & $ k $ & $ k' $ \\
\hline
\end{tabular}
\end{table}
Expanding \cref{eq:altsum} according to \cref{t:iter} we find,
\begin{align}
\frac{1}{F}\left\{\begin{aligned}
& (n_{1} - n_{2})(H_{11} - H_{22}) + (n_{1} - n_{3})(H_{11} - H_{33}) + \dots + (n_{1} - n_{F})(H_{11} - H_{FF}) + \\
& (n_{2} - n_{3})(H_{22} - H_{33}) + (n_{2} - n_{4})(H_{22} - H_{44}) + \dots + (n_{2} - n_{F})(H_{22} - H_{FF}) + \dots + \\
& (n_{F-1} - n_{F})(H_{(F-1)(F-1)} - H_{FF})
\end{aligned}\right\}~.
\end{align}
Which can be rearranged as \cref{eq:altsumr}
\begin{align}\label{eq:altsumr}
\frac{1}{F}\left\{\begin{aligned}
& \left[ (F-1)n_{1}H_{11} - n_{2}H_{11} - \dots - n_{F}H_{11} \right] + \\
& \left[ - n_{1}H_{22} + (F-1)n_{2}H_{22} - \dots - n_{F}H_{22} \right] + \dots + \\
& \left[ - n_{1}H_{FF} - n_{2}H_{FF} - \dots + (F-1)n_{F}H_{FF} \right]
\end{aligned}\right\}~,
\end{align}
and simplified as,
\begin{align}\label{eq:rear2}
\frac{1}{F} \left\{\begin{aligned}
& H_{11} \left[ (F-1) n_{1} - n_{2} - \dots - n_{F} \right] + \\
& H_{22} \left[ - n_{1} + (F-1) n_{2} - \dots - n_{F} \right] + \dots + \\
& H_{FF} \left[ - n_{1} - n_{2} - \dots + (F-1) n_{F} \right]
\end{aligned}\right\}~.
\end{align}
Which is exactly the same as \cref{eq:rear}, and therefore,
\begin{align}
\frac{1}{F}\sum\limits_{k}^{F} n_{k} (H_{kk} - \bar{H}) = \frac{1}{F} \sum\limits_{k<k'=1}^{F} (n_{k}-n_{k'})(H_{kk}-H_{k'k'})~.
\end{align}
Which means our Hamiltonian can be written as,
\begin{align}\label{eq:almost}
H(\bm{P}, \bm{R}, \bm{p}, \bm{x}) & =\frac{\bm{P}^{2}}{2\mu}+\bar{H}(\bm{R}) 
\nnum
& +\sum\limits_{k<k'=1}^{F}
\left\{\begin{aligned}
\frac{1}{F} (H_{kk}(\bm{R})-H_{k'k'}(\bm{R}))&\cdot\left(n_{k}-n_{k'}\right)\\
+H_{kk'}(\bm{R})&\cdot(p_{k}p_{k'}+x_{k}x_{k'})
\end{aligned}\right\}~,
\end{align}
and using \cref{eq:nk}, \cref{eq:almost} can be fully expressed in cartesian coordinates,
\begin{align}
H(\bm{P}, \bm{R}, \bm{p}, \bm{x}) & =\frac{\bm{P}^{2}}{2\mu}+\bar{H}(\bm{R}) 
\nnum
& +\sum\limits_{k<k'=1}^{F}
\left\{\begin{aligned}
\frac{1}{F} (H_{kk}(\bm{R})-H_{k'k'}(\bm{R}))&\cdot\left(
\frac{1}{2}p_{k}^{2}+\frac{1}{2}x_{k}^{2}-\frac{1}{2}p_{k'}^{2}-\frac{1}{2}x_{k'}^{2}\right)\\
+H_{kk'}(\bm{R})&\cdot(p_{k}p_{k'}+x_{k}x_{k'})
\end{aligned}\right\}~.
\end{align}
\begin{center}\boxed{QED}\end{center}
%\chapter{Notes}
%\section{Week 0-1}
%\includepdf[pages={1-8}]{thesis_notes1_celis.pdf}