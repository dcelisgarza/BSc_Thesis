\chapter{Conclusions}\label{c:conc}
%
\section{Conclusion}
%
The most recent version of the MM model was successfully implemented in \textsc{fortran 2008}. It was validated with the three avoided crossing problems defined by Tully \cite{tully}, and subsequently analysed by Cotton and Miller \cite{project}. Aside from some minor discrepancies attributed to integration step size, RNG, compiler, and number of trajectories used, our results show very good agreement with those found in \cite{project}.

Furthermore, new results were obtained for all three avoided crossing problems, where $ i = 2 $, as well as new parameter sets for the spin-boson model for $ i = 1, 2 $. 

In the case of the three avoided crossing problems, these new results should be verified using a true quantum mechanical model. This is uncharted territory, given that the systems' initial states are excited, rather than ground states. However, it is not unreasonable to assume that these new results are accurate, or at least, as accurate as the results for the ground state. The reason why is because the MM model is a continuous model, so even when $ i = 1 $, the particle can smoothly move out of the ground state and into the excited one; and its trajectory will smoothly and continuously evolve just as before. So there is no reason why having $ i = 2 $, should be any different than back-tracing the trajectory of a particle with $ f = 2 $.

As mentioned in \cref{c:r}, the results of the spin-boson model problems are surprising. We were fortunate to see such a wide variety of behaviours with our choice of parameters. Not only did we see coherent and incoherent oscillations, but we saw the radically different behaviours between the symmetric and asymmetric versions of all problems. The reason for this difference is simple---the energy difference between both states widens, and therefore we get very anisotropic\footnote{When a forward process is not the same as its reverse. For example, flow valves, they allow flow in one direction but not the reverse. Mathematically this can be written as $ A \rightarrow B \neq A \leftarrow B$. In the present case this is taken to mean that if $ x $ is the probability of transitioning from $ A \rightarrow B $, then $ A \leftarrow B $ is drastically different from $ x $ and not necessarily in the neighbourhood of $ 1-x $.} transition probabilities. The other surprising (and new) results are the large stability regions observed in one parameter set. Such results, make this parameter set a prime candidate for the study of quantum tunnelling in large systems. Unfortunately, this is non-trivial, because it requires a model which takes quantum tunnelling into consideration.

Lastly, the code's performance and scaling were found to be excellent. Its performance under parallelisation was found to be much better than expected; the results are not affected whatsoever, and the computational time is inversely proportional to the number of cores---this is exactly what we want to see. All that remains in this regard is to parallelise the code itself, rather than manually have to run different copies in different cores and averaging the results.
%
\section{Future Work}
%
In order to identify the source of the discrepancies, efforts should be made to lightly modify the code by using different RNGs and seed generators, as well as doing further testing with smaller integration intervals and more MC reps. The code should also be generalised so it can read input files and be applied to arbitrary systems. The use of an adaptive integration algorithm could also be worth investigating. Lastly, attempts must be made to eliminate the need for analytic PES so the model may be applied to real systems. If the code can be sufficiently generalised, it will be added to \textsc{cfour} \cite{cfour}, and thus be freely placed at the disposition of the scientific community.
%